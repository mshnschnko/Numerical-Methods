\documentclass[a4paper,14pt]{report}

\usepackage{cmap}
\usepackage[T2A]{fontenc}
\usepackage[utf8]{inputenc}
\usepackage[english,russian]{babel}
\usepackage{setspace}
\usepackage{geometry}
\geometry{top=2cm}
\geometry{bottom=2cm} % отступ снизу
\geometry{left=2cm} % отступ справа
\geometry{right=2cm} % отступ слева
\usepackage{graphicx}
\graphicspath{{./images/}}
\makeatletter %%%%% <---- Starting chapter without a pagebreak
\renewcommand\chapter{\par%
	\thispagestyle{plain}% \global\@topnum\z@
	\@afterindentfalse \secdef\@chapter\@schapter}
\makeatother

\begin{document}
	
	\begin{center}
		\hfill \break
		\begin{center}
		\huge{Санкт-Петербургский политехнический университет\\
			Высшая школа прикладной математики\\
			и вычислительной физики, ФизМех}
		\end{center}
		\hfill \break
		\hfill \break
		\hfill \break
		\hfill \break
		\hfill \break

		\huge{Направление подготовки\\
		<<Прикладная математика и информатика>>}\\
		\hfill \break
		\hfill \break
		\hfill \break
		\hfill \break
		\hfill \break
		\hfill \break
		\fontsize{14pt}{14pt}\selectfont
		Отчет по лабораторной работе №1\\
			<<Решение алгебраических и трансцендентных уравнений>>\\
		\begin{center}
			Дисциплина: <<Численные методы>>
		\end{center}
		\hfill \break
		\hfill \break
		\hfill \break
		\hfill \break
		\hfill \break
	\end{center}
	\hfill \break
	\hfill \break
	\fontsize{12pt}{12pt}\selectfont 
		\begin{tabular}{cccc}
			\hspace{1cm}Выполнил студент гр. 5030102/00003 & {\hspace{3cm}} & & Анищенко М.Д. \\\\
			\hspace{-3cm}Преподаватель: &{\hspace{1cm}}& & {\hspace{1cm}} Курц В.В. \\\\
		\end{tabular}\\
	\hfill \break
	\hfill \break
	\hfill \break
	\hfill \break
	\hfill \break
	\hfill \break
	\begin{center} Санкт-Петербург 2021 \end{center}
	\thispagestyle{empty}
	
	\newpage
	\tableofcontents{}
	\clearpage
	
	\chapter{Формулировка задачи и её формализация}
	Существует два типа методов решения уравнений: аналитические и численные. На практике часто приходится сталкиваться с уравнениями, которые невозможно или затруднительно решит аналитически.\\
	Основные проблемы аналитических методов:\\
	\begin{itemize}
		\item Алгебраические уравнения разрешимы только до 4-ой степени
		\item Трансцендентные уравнения в общем случае неразрешимы
		\item Заранее неизвестно, сколько корней и существуют ли они вообще
	\end{itemize}
	\section{Формализация задачи}
	Пусть есть \(f(x):R \rightarrow R\) – алгебраическая или трансцендентная функция одной переменной. Требуется найти такой \(x^{*}\), что \(f(x^{*})=0\) с помощью двух численных методов и сравнить их эффективность.
	
	\section{Постановка задачи}
	\begin{enumerate}
		\item Исследовать условия применимости методов и алгоритм их работы
		\item Решить алгебраическую и трансцендентную функции с помощью метода половинного деления и комбинированного метода хорд и касательных
		\item Построить графики для анализа результатов и сравнить решения с функцией fzero в MATLAB
	\end{enumerate}

	\chapter{Алгоритмы методов и условия их применимости}
	\section{Метод половинного деления}
	\subsection{Условия применимости}
	\begin{enumerate}
		\item \(f \in C([a,b])\)
		\item \(f(a)f(b)<0\)
	\end{enumerate}
	\subsection{Алгоритм}
	Для выбранного промежутка \([a,b]\) принимаем \(c=\frac{a+b}{2}\). Вычисляем \(f(a) \cdot f(c)\). Если значение этого выражения меньше нуля, то принимаем \(b=c\), иначе \(a=c\). Повторяем эти действия, пока \(|b-a| \leq 2\varepsilon\). В качестве конечного ответа возмём \(x=\frac{a+b}{2}\).
	
	\section{Комбинированный метод хорд и касательных (Ньютона)}
	\subsection{Условия применимости метода хорд}
	\begin{enumerate}
		\item \(f \in C^{(2)}([a,b])\)
		\item \(f(a)f(b)<0\)
		\item \(f', f''\) знакопостоянны на \([a,b]\)
		\item Стартовая точка \(x^{(0)}:f(x^{(0)})f''(x^{(0)})<0\)
		\item Неподвижный конец \(\bar{x}:f(\bar{x})f''(\bar{x})>0\)
	\end{enumerate}
	%Тогда последовательность \[ x^{(k+1)} = x^{(k)}-f(x^{(k)}) \frac {x^{(k)}-\bar{x}}{f(x^{(k)})-f(\bar{x})}\] сходится.
	
	\subsection{Алгоритм метода хорд}
	Данный метод является итерационным. Для начала выберем стартовую и неподвижную точки, в качестве которых можно взять концы отрезка \([a,b]\) так, чтобы в этих точках выполнялись условия 4 и 5 соответственно. Далее действуем по итерационной формуле: \[ x^{(k+1)} = x^{(k)}-f(x^{(k)}) \frac {x^{(k)}-\bar{x}}{f(x^{(k)})-f(\bar{x})}\] где \(m_1=min|f'(x)|\) и \(M_1=max|f'(x)|\), \(x \in [a,b]\)
	
	Для остановки этого процесса существует такая оценка: \[|x^{*}-x^{k+1}| \leq \frac{M_1-m_1}{m_1}|x^{k+1}-x^{k}|\]
	
	\subsection{Условия применимости метода касательных}
	\begin{enumerate}
		\item \(f \in C^{(2)}([a,b])\)
		\item \(f(a)f(b)<0\)
		\item \(f', f''\) знакопостоянны на \([a,b]\)
		\item \(x^{(0)}:f(x^{(0)})f''(x^{(0)})>0\) (условие Фурье)
	\end{enumerate}
	%\begin{spacing}{1.6}
		%\fontsize{12pt}{12pt}\selectfont 
		%Тогда последовательность \(\{x^{(k)}\}_{k=0}^{\infty}\), построенная по методу Ньютона, монотонно сходится к корню \(x^{*} \in (a,b)\).
	%\end{spacing}

	\subsection{Алгоритм метода касательных}
	Метод касательных (Ньютона) так же итерационный. В качестве стартовой точки выбирается один из концов отрезка \([a,b]\) так, чтобы выбранная точка удовлетворяла условию Фурье. Затем с помощью разложения функции \(f\) в ряд Тейлора в окрестности \(x^{(k)}\) и дальнейшего отбрасывания слагаемых второй и выше степеней получаем итерационную формулу: \[x^{(k+1)} = x^{(k)}- \frac{f(x^{(k)})}{f'(x^{(k)})}\]
	Итерационный процесс можно останавливать, когда \(|x^{(k+1)}-x^{(k)}|<\varepsilon\).
	
	\subsection{Алгоритм комбинированного метода}
	Применяя оба итерационных метода на каждом шаге нахождение корня ускоряется, так как приближение к нему происходит с двух сторон. В таком случае можно находить корень без использования оценки погрешности, сужая интервал как в методе половинного деления. При это мнадо учесть, что неподвижная точка в методе хорд будет меняться. Следующие значения концов сужающего отрезка выбираются согласно такому алгоритму:
	\fontsize{14pt}{14pt}\selectfont 
	\begin{itemize}
		\item Если \(f(a)f''(a)<0\), то \(a=a-f(a)\frac{a-b}{f(a)-f(b)}\)\\
		иначе если \(f(a)f''(a)>0\), то \(a=a-\frac{f(a)}{f'(a)}\)
		\item Если \(f(b)f''(b)<0\), то \(b=b-f(b)\frac{b-a}{f(b)-f(a)}\)\\
		иначе если \(f(b)f''(b)>0\), то \(b=b-\frac{f(b)}{f'(b)}\)
	\end{itemize}
	\fontsize{12pt}{12pt}\selectfont 
	Заканчиваем итерационный процесс, когда \(|b-a|<2\varepsilon\)\\
	В качестве корня берем значение \[x= \frac{b+a}{2}\]
	
	\newpage
	\chapter{Анализ задачи}
	\section{Алгебраическая функция}
	Возьмём функцию \[f(x) = 2x^5+3x^4 - 4x^3 - 12x^2+10\]
	\subsection{Границы корней}
	\begin{itemize}
		\item Верхняя граница положительных корней \[x^* \leq 1+ \sqrt[m]{\frac{|a'|}{a_0}}\]\\
		Для данной функции \[x^* \leq 1+\sqrt[2]{\frac{12}{2}} \approx 3.45\]
		\item Нижняя граница положительных корней\\
		Выполним замену \(x=\frac{1}{y}\)
		\[y \leq 1 + \sqrt[1]{\frac{12}{10}}=2.2\]\\
		\[x^* \geq \frac{1}{2.2} \approx 0.45\]
		\item Нижняя граница отрицательных корней\\
		Выполним замену \(x=-y\)
		\[y \leq 1+\sqrt[1]{\frac{10}{2}}=6\]
		\[x^* \geq -6\]
		\item Верхняя граница отрицательных корней\\
		Выполним замену \(x=-\frac{1}{y}\)
		\[y \leq 1 + \sqrt[1]{\frac{12}{10}}=2.2\]\\
		\[x^* \leq -\frac{1}{2.2} \approx -0.45\]
	\end{itemize}
	Тогда для положительных корней \(x^* \in [0.45, 3.45]\), для отрицательных \(x^* \in [6, -0.45]\).
	\subsection{Проверка выполнения условий применимости}
	Воспользуемся теоремой о границах корней и графическим методом. Заметим, что найденный для положительных корней отрезок содержит два корня. Поэтому сузим промежуток до \([1.3, 3]\).
	\begin{itemize}
		\item Метод половинного деления:
		{
			\begin{enumerate}
				\item \(f \in C([1.3, 3])\)
				\item \(f(1.3)f(3)<0\),  (\((-3.0738)\cdot595<0\))
			\end{enumerate}
		}
		\item Комбинированный метод:
		{
			\begin{enumerate}
				\item \(f \in C^{(2)}([1.3, 3])\)
				\item \(f(1.3)f(3)<0\),  (\((-3.0738)\cdot595<0\))
				\item \(f', f''\) знакопостоянны на \([1.3, 3]\)
				\item Стартовая точка для метода хорд \(x^{(0)}=a=1.3\), \(f(1.3)f''(1.3)=\\=(-3.0738)\cdot93.52<0\)
				\item Неподвижный (начальный конец) \(\bar{x}=b=3\), \(f(3)f''(3)=595\cdot1308>0\)
				\item Стартовая точка для метода касательных \(x^{(0)}=b=3\), вычисления указаны выше
			\end{enumerate}
		}
	\end{itemize}
	Условия применимости проверены для обоих методов, следовательно выбранный отрезок подходит для вычисления корня на нём.
	
	\section{Трансцендетная функция}
	Возьмём функцию \[g(x)=5^{x}-\sin(x)-3\]
	Используя графический метод представления, заметим, что функция имеет единственный корень на отрезке \([0.8, 2]\).
	\begin{itemize}
		\item Метод половинного деления:
		{
			\begin{enumerate}
				\item \(f \in C([0.8, 2])\)
				\item \(f(0.8)f(2)<0\),  (\((-0.0935)\cdot21.0907<0\))
			\end{enumerate}
		}
		\item Комбинированный метод:
		{
			\begin{enumerate}
				\item \(f \in C^{(2)}([0.8, 2])\)
				\item \(f(0.8)f(2)<0\),  (\((-0.0935)\cdot21.0907<0\))
				\item \(f', f''\) знакопостоянны на \([0.8, 2]\)
				\item Стартовая точка для метода хорд \(x^{(0)}=a=0.8\), \(f(0.8)f''(0.8)=\\=(-0.0935)\cdot10.1043<0\)
				\item Неподвижный (начальный конец) \(\bar{x}=b=2\), \(f(2)f''(2)=\\=21.0907\cdot65.6665>0\)
				\item Стартовая точка для метода касательных \(x^{(0)}=b=2\), вычисления указаны выше.
			\end{enumerate}
		}
	\end{itemize}
	Условия применимости проверены для обоих методов, следовательно выбранный отрезок подходит для вычисления корня на нём.
	\newpage
	\chapter{Тестовый пример}
	В качестве тестового примера возмём простое квадратное уравнение \(x^2+x-6=0\). Нетрудно определить, что корни данного уравнения \(x_{1,2}=-3; 2\). Выберем для поиска корня отрезок \([1, 4]\). Зададим точность \(\varepsilon=0.01\).
	\section{Метод половинного деления}
		Проверим выполнение условий применимости:
	\begin{enumerate}
		\item \(f\) непрерывна на \([1, 4]\)
		\item \(f(1)f(4)<0\), \((-4)\cdot14<0\)
	\end{enumerate}
	Этапы нахождения корня:
	\begin{enumerate}
		\item \([1, 2.5]\) \hspace{1.5cm} \(f(a) = -4\) \hspace{1.5cm} \(f(b) = 2.75\) \hspace{1.5cm} \(f(c) = 2.75\) \hspace{1.5cm} \(c = 2.5\)
		\item \([1.75, 2.5]\) \hspace{1cm} \(f(a) = -1.1875\) \hspace{1cm} \(f(b) = 2.75\) \hspace{1cm} \(f(c) = -1.1875\) \hspace{1cm} \(c = 1.75\)
		\item \([1.75, 2.125]\) \hspace{0.5cm} \(f(a) = -1.1875\) \hspace{0.5cm} \(f(b) = 0.640625\) \hspace{0.5cm} \(f(c) = 0.640625\) \hspace{0.5cm} \(c = 2.125\)
		\item \([1.9375, 2.125]\)\hspace{0.5cm}\(f(a) = -0.308594\)\hspace{0.5cm}\(f(b) = 0.640625\)\hspace{0.5cm}\(f(c) = -0.308594\)\hspace{0.5cm}\(c = 1.9375\)
		\item \([1.9375, 2.03125]\)\hspace{0.5cm}\(f(a) = -0.308594\)\hspace{0.5cm}\(f(b) = 0.157227\)\hspace{0.5cm}\(f(c) = 0.157227\)\hspace{0.5cm}\(c=2.03125\)
		\item \([1.984375, 2.03125]\) \(f(a) = -0.077881\) \(f(b) = 0.157227\) \(f(c) = -0.077881\) \(c=1.984375\)
		\item \([1.984375, 2.007813]\) \(f(a) = -0.077881\) \(f(b) = 0.039124\)	\(f(c) = 0.039124\)	\(c=2.007813\)
		\item \([1.996094, 2.007813]\) \(f(a) = -0.019516\) \(f(b) = 0.039124\) \(f(c) = -0.019516\) \(c=1.996094\)
	\end{enumerate}
	\[x=\frac{a+b}{2}=\frac{1.996094+2.007813}{2}=2.0019535\]
	\[|x^*-x|=|2-2.0019535|=0.0019535\approx0.002\]
	Корень был найден с абсолютной погрешностью \(\delta = 0.002\) при заданной точности \(\varepsilon=0.01\) за 8 итераций, то есть данное решение удовлетворяет нашим требованиям.
	
	\section{Комбинированный метод}
	Проверим выполнение условий применимости:
	\begin{enumerate}
		\item \(f\) непрерывна на \([1, 4]\)
		\item \(f(1)f(4)<0\), \((-4)\cdot14<0\)
		\item \(f',f''\) знакопостоянны на \([1, 4]\)
		\item Стартовая точка для метода хорд \(x^{(0)}=a=1\), \(f(1)f''(1)=(-4)\cdot2<0\)
		\item Неподвижный (начальный) конец \(\bar{x}=b=4\), \(f(4)f''(4)=14\cdot2>0\)
		\item Стартовая точка для метода касательных \(x^{(0)}=b=4\), вычисления указаны выше.
	\end{enumerate}
	Этапы нахождения корня:
	\begin{enumerate}
		\item \([2.333333, 2.444444]\) \hspace{2.5cm} \(f(a)=1.777778\) \hspace{2.5cm} \(f(b)=2.419753\)
		\item \([2.019608, 2.033543]\) \hspace{2.5cm} \(f(a)=0.098424\) \hspace{2.5cm} \(f(b)=0.16884\)
		\item \([2.000076, 2.000222]\) \hspace{2.5cm} \(f(a)=0.000381\) \hspace{2.5cm} \(f(b)=0.00111\)
	\end{enumerate}
	\[x=\frac{a+b}{2}=\frac{2.000076+2.000222}{2}=2.000149\]
	\[|x^*-x|=|2-2.000149|=0.000149\approx0.0001\]
	Корень был найден с абсолютной погрешностью \(\delta=0.0001\) при заданной точности \(\varepsilon=0.01\) за 3 итерации, то есть данное решение удовлетворяет нашим требованиям.\\
	
	Заметим, что с помощью комбинированного метода ответ был получен быстрее и с большей точностью, следовательно комбинированный метод хорд и касательных эффективнее метода половинного деления.
	
	\chapter{Контрольные тесты}
	Найдём по одному корню у выбранных алгебраической и трансцендентной функций двумя методами.
	\begin{itemize}
		\item \(f(x)=2x^5+3x^4-4x^3-12x^2+10\), \(x \in [1.3, 3]\), будем изменять точность \(\varepsilon\) от \(10^{-10}\) до \(10^{-2}\), увеличивая $\varepsilon$ в 10 раз на каждом шаге. Также будем изменять отрезок \([a, b]\) от \([1.3, 3]\) до \([1.3, 30]\) с шагом \(\Delta=3\) для фиксированной точности \(\varepsilon=10^{-7}\).
		\item \(g(x)=5^x-\sin(x)-3\), \(x\in[0.8, 2]\), будем изменять точность так же от \(10^{-10}\) до \(10^{-2}\), увеличивая $\varepsilon$ в 10 раз на каждом шаге, а отрезок \([a, b]\) от \([0.8, 2]\) до \([0.8, 29]\) с шагом \(\Delta=3\) для фиксированной точности \(\varepsilon=10^{-7}\).
	\end{itemize}
	\newpage
	\chapter{Модульная структура программы}
	\section{Заголовки функций и их значение}
	\begin{enumerate}
		\item typedef struct root --- структура, хранящая значение корня и номер итерации для соответствующего значения.
		\item double FuncAlg(double x); --- функция для нахождения значения алгебраической функции в данной точке.
		\item double FuncTrans(double x); --- функция для нахождения значения трансцендентной функции в данной точке.
		\item double df(double x, double (*f)(double)); --- функция, вычисляющая значение производной в данной точке. На вход поступает точка $x$ и указатель на функцию (алгебраическую или трансцендентную).
		\item double d2f(double x, double (*f)(double)); --- функция, вычисляющая значение второй производной в данной точке.
		\item root Bisection(double a, double b, double e, double (*f)(double), const char* filename, const char* filenameones) --- функция, вычисляющая корень с помощью метода половинного деления. На вход поступают границы отрезка \([a, b]\), заданная точность $\varepsilon$, указатель на математическую функцию (алгебраическую или трансцендентную), имена файлов для записи информации об этапах нахождения корней.
		\item root ChordNewton(double a, double b, double e, double (*f)(double), const char* filename, const char* filenameones); --- функция, вычисляющая корень с помощью комбинированного метода хорд и касательных. Входные данные аналогичные предыдущему пункту.
	\end{enumerate}
	\newpage
	\chapter{Численный анализ}
	Графики функций:\\
	\begin{center}
		\includegraphics{functions}\\
	\end{center}
	\section{Вычисление корней}
	Для заданной точности \(\varepsilon=10^{-7}\) получились следующие результаты:
	\begin{itemize}
		\item Алгебраическая функция
		\begin{enumerate}
			\item МПД: \(x=1.5043907\) (24 итерации).
			\item Комбинированнный: \(x=1.5043907\) (8 итераций).
		\end{enumerate}
		\item Трансцендентная функция
		\begin{enumerate}
			\item МПД: \(x=0.8178801\) (23 итерации).
			\item Комбинированный: \(x=0.8178801\) (6 итераций).
		\end{enumerate}
	\end{itemize}
	Видим, что корни, найденные двумя разными методами, совпали. При этом более эффективным оказался комбинированный метод (быстрее в 3--4 раза).
	\section{Сравнение с fzero() из MATLAB}
	С помощью пакета MATLAB было произведено сравнение, которое показало, что погрешность корней, найденных данными методами и с помощью fzero, имеет порядок \(10^{-8}\) для МПД и \(10^{-12}\) для комбинированного метода. Значит, корни были найдены верно.\\
	
	Достижение точности:\\
	
	\begin{center}
		\includegraphics{accuracy}\\
	\end{center}
	\section{Сходимость}
	\begin{center}
		\includegraphics{converenge}\\
	\end{center}
	
	Метод половинного деления имеет линейную сходимость, однако сходится немонотонно.\\
	Определим порядок сходимости комбинированного метода, зная \(|x^*-x^{(k+1)}| \leq C|x^*-x^{(k)}|^p\). С помощью пакета MATLAB прологарифмируем выражение и найдём $p$. Видим, что $p$ принимает значение около 5. При этом комбинированный метод сходится монотонно.
	\section{Объём вычислений}
	\begin{center}
		\includegraphics{calcamount}\\
	\end{center}
	Метод половинного деления имеет линейную зависимость количества итераций от заданной точности. Количество итераций комбинированного метода возрастает гораздо медленнее.
	\section{Влияние удалённости стартовой точки}
	\begin{center}
		\includegraphics{influence}
	\end{center}
	По графикам видно, что метод половинного деления слабо зависим от удалённости стартовой точки. Это можно отметить для обеих функций. Комбинированный метод ведет себя хуже. Особенно это заметно для трансцендентной функции.
	\chapter{Общие выводы}
	\hspace{0.5cm}В ходе данной работы мы научились вычислять корни алгебраических и трансцендентных функций с помощью методов половинного деления и комбинированного, сравнили их скорости сходимости, объёмы вычислений, влияние удалённости начальной точки. Мы выяснили, что МПД стабилен для любых функций и стартовых точек и линейно приближается к корню. Комбинированный метод сильно зависит от самой функции и может вести себя менее предсказуемо.
	
	Как итог, мы видим, что для использования комбинированного метода необходимо детально исследовать функцию. В таком случае он окажется более эффективным. Если же детальное исследование функции не производится, то имеет место использование МПД.
\end{document}
