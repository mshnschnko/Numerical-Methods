\documentclass[a4paper,14pt]{report}

\usepackage{cmap}
\usepackage[T2A]{fontenc}
\usepackage[utf8]{inputenc}
\usepackage[english,russian]{babel}
\usepackage{setspace}
\usepackage{geometry}
\geometry{top=2cm}
\geometry{bottom=2cm} % отступ снизу
\geometry{left=2cm} % отступ справа
\geometry{right=2cm} % отступ слева
\usepackage{graphicx}
\usepackage{amsmath}
\graphicspath{{./images/}}
\makeatletter %%%%% <---- Starting chapter without a pagebreak
\renewcommand\chapter{\par%
	\thispagestyle{plain}% \global\@topnum\z@
	\@afterindentfalse \secdef\@chapter\@schapter}
\makeatother

\begin{document}
	
	\begin{center}
		\hfill \break
		\begin{center}
			\huge{Санкт-Петербургский политехнический университет\\
				Высшая школа прикладной математики\\
				и вычислительной физики, ФизМех}
		\end{center}
		\hfill \break
		\hfill \break
		\hfill \break
		\hfill \break
		\hfill \break
		
		\huge{Направление подготовки\\
			<<Прикладная математика и информатика>>}\\
		\hfill \break
		\hfill \break
		\hfill \break
		\hfill \break
		\hfill \break
		\hfill \break
		\fontsize{14pt}{14pt}\selectfont
		Отчет по лабораторной работе №2\\
		<<Решение СЛАУ прямыми методами>>\\
		\begin{center}
			Дисциплина: <<Численные методы>>
		\end{center}
		\hfill \break
		\hfill \break
		\hfill \break
		\hfill \break
		\hfill \break
	\end{center}
	\hfill \break
	\hfill \break
	\fontsize{12pt}{12pt}\selectfont 
	\begin{tabular}{cccc}
		\hspace{1cm}Выполнил студент гр. 5030102/00003 & {\hspace{3cm}} & & Анищенко М.Д. \\\\
		\hspace{-3cm}Преподаватель: &{\hspace{1cm}}& & {\hspace{1cm}} Курц В.В. \\\\
	\end{tabular}\\
	\hfill \break
	\hfill \break
	\hfill \break
	\hfill \break
	\hfill \break
	\hfill \break
	\begin{center} Санкт-Петербург 2021 \end{center}
	\thispagestyle{empty}
	
	\newpage
	\tableofcontents{}
	\clearpage
	\chapter{Формулировка задачи и её формализация}
	Большинство расчётных математических задач сводится к решению систем линейных алгебраических уравнений. Существует два типа методов решения таких задач: прямые и итерационные. В данной работе будет рассмотрен прямой метод, т.е. метод, который приводит к точному, а не приближенному ответу за конечное число операций, зависящее только от количества неизвестных.
	\section{Формализация задачи}
	Пусть дана система из n линейных уравнений с n неизвестными. \(Ax=B\) -- её матричная форма записи. Требуется найти вектор X, удовлетворющий данному матричному уравнению и являющийся решением данной СЛАУ. В этой работе буде использован метод прогонки (Thomas Algorithm).
	\section{Постановка задачи}
	\begin{enumerate}
		\item Исследовать условия применимости метода и алгоритм его работы.
		\item Решить СЛАУ данным методом.
		\item Построить графики для анализа выбранного метода, а именно: его поведение при разных обусловленности матрицы, её размерах и возмущении правой части.
	\end{enumerate}
	\chapter{Алгоритм метода и условия его применимости}
	\section{Условия применимости метода}
	Метод прогонки применяется для трёхдиагональных матриц. Также метод прогонки будет корректным и устойчивым, если коэффициенты матрицы A	удовлетворяют условиям диагонального преобладания:
	\[|c_i|>|b_i|+|d_i|\hspace{1em}\forall i=1,\ldots, n\]
	\section{Алгоритм метода}
	Пусть дана трёхдиагональная матрица. Сделаем следующие обозначения:
	$$A=
	\begin{bmatrix}
		c_1& d_1& 0& \cdots& 0& 0\\
		b_2& c_2& d_2& \cdots& 0& 0\\
		0& b_3& c_3& \cdots& 0& 0\\
		\vdots& \vdots& \vdots& \ddots& \vdots& \vdots\\
		0& 0& 0& \cdots& c_{n-1}& d_{n-1}\\
		0& 0& 0& \cdots& b_n& c_n\\
	\end{bmatrix}
	\begin{bmatrix}
		x_1\\
		x_2\\
		x_3\\
		\vdots\\
		x_{n-1}\\
		x_n\\
	\end{bmatrix}
	=
	\begin{bmatrix}
		r_1\\
		r_2\\
		r_3\\
		\vdots\\
		r_{n-1}\\
		r_n\\
	\end{bmatrix}
	$$
	Причем \(b_1=0, d_n=0\).\\
	Пусть \(\exists \delta_i, \lambda_i:\)
	\begin{equation}
		x_i=\delta_i x_{i+1}+\lambda_i, i=1,2,\ldots,n
		\label{x_rec}
	\end{equation}
	Рекурентные выражения для \(\delta_i\) и \(\lambda_i\):
	\begin{equation}
		\delta_i=-\frac{d_i}{b_i \delta_{i-1}+c_i}, \lambda_i=\frac{r_i-b_i \lambda_{i-1}}{b_i\delta_{i-1}+c_i}, i=1,2,\ldots,n
		\label{dl_rec}
	\end{equation}
	\begin{enumerate}
		\item Прямой ход: используя формулу~\eqref{dl_rec}, находим значения \(\delta_1,\ldots, \delta_n\) и \(\lambda_1, \ldots, \lambda_n\), принимая \(i=1,\ldots,n\)
		\item Обратный ход: используя формулу~\eqref{x_rec}, находим значения \(x_n,\ldots,x_1\), принимая \(i=n,\ldots,1\)
	\end{enumerate}
	\chapter{Анализ задачи}
	Чтобы построить матрицу \(A\) с разным заданным числом обусловленности, зададим её как \(A=QDQ^T\), где \(Q\) -- ортогональная матрица, \(D\) -- диагональная, которую мы составили так, чтобы \(cond(A)=\frac{|\lambda_{max}|}{|\lambda_{min}|}\), где \(\lambda_i\) -- собственное число. Затем приведём полученную матрицу \(A\) к форме Хессенберга и получим трёхдиагональную матрицу с тем же числом обусловленности.
	\newpage
	\chapter{Тестовый пример}
	В качестве тестового примера возьмём следующую систему линейных алгебраических уравнений:
	$$
	\begin{cases}
		2x_1+x_2=4\\
		-x_1+5x_2+3x_3=-3,\\
		4x_2-2x_3=16\\
	\end{cases}
	A =
	\begin{pmatrix}
		2& 1& 0\\
		-1& 5& 3\\
		0& 4& -2\\
	\end{pmatrix},
	B=
	\begin{pmatrix}
		4\\
		-3\\
		16\\
	\end{pmatrix}
	$$
	Найдём числа \(\delta_1, \delta_2, \delta_3\) и \(\lambda_1, \lambda_2, \lambda_3\), применив прямой ход:
	
	$$\delta_1 = -\frac{1}{2} = -0.5, \lambda_1 = \frac{4}{2} = 2$$
	$$\delta_2 = -\frac{3}{-1\cdot (-0.5) + 5} = -\frac{6}{11} = -0.5454545454, \lambda_2 = \frac{-3-(-1)\cdot 2}{-1\cdot (-0.5) + 5} = -\frac{2}{11} = -0.1818181818$$
	$$\delta_3 = 0, \lambda_3 = \frac{16-4\cdot (-0.1818181818)}{4\cdot (-0.5454545454)+(-2)} = -4.0000000002$$
	
	Теперь найдём числа \(x_3, x_2, x_1\), применив обратный ход:
	$$x_3 = -4.0000000002 \approx -4$$
	$$x_2 = -0.5454545454\cdot (-4.0000000002) +(-0.1818181818) = 1.99999999990909 \approx 2$$
	$$x_1 = -0.5\cdot 1.99999999990909 + 2 = 1.000000000045455 \approx 1$$
	Как мы можем видеть, метод находит решение за число шагов, пропорциональное количеству неизвестных. Погрешность ответа появилась из-за дроби, не приводимой к конечному десятичному виду. Если подставлять в выражения простые дроби, то ответ получится точным.
	\chapter{Контрольные тесты}
	Создадим 10 матриц размером 15$\times$15 с числами обусловленности от 10 до $10^{10}$ и для каждой найдём корни методом прогонки.
	Так же для оценки метода будем вносить в правую часть возмущение порядков от \(10^{-5}\) до 1.
	Для оценки времени выполнения метода будет менять размерность матрицы от 15$\times$15 до 500$\times$500.
	\newpage
	\chapter{Модульная структура программы}
	\begin{itemize}
		\item double** CreateMatrix(int n, int m); --- функция, которая выделяет память под матрицу с заданным количеством строк и столбцов.
		\item void DestroyMatrix(double** A, int n); --- функция, которая очищает память, выделенную под матрицу.
		\item double** ReadMatrix(FILE* matrixfile, FILE* freekoeffile, int n); --- функция, выполняющая считывание матриц с файла.
		\item void PrintVector(FILE* rootsfileforrec, double* arr, int n); --- функция, выполняющая запись вектора корней в файл.
		\item double* Thomas(double** matrix, int n, FILE* time); --- функция, осуществляющая непосредственно расчёты по методу прогонки.
	\end{itemize}
	\chapter{Численный анализ}
	\section{Влияние обусловленности матрицы}
	Для того, чтобы узнать, как обусловленность матрицы влияет на конечный ответ, выполним следующие действия. Сравним относительную погрешность нормы полученного с помощью выбранного метода ответа и заданным вектором X. Построим график зависимости относительной погрешности от числа обусловленности с помошью пакета MATLAB.
	\begin{center}
		\includegraphics{CondVal}
	\end{center}
	Мы видим, что относительная погрешность ответа растёт при увеличении числа обусловленности, как и ожидалось. При этом точность при наилучшей обусловленности достигает порядка \(10^{-15}\), что близко к машинному $\varepsilon$, т.е. ответ можно назвать точным.
	\section{Влияние возмущения правой части}
	Внесём возмущения в правую часть, согласно выбранному диапазону (Глава 5). Так же, как и в предыдущем пункте, воспользуемся пакетом MATLAB, чтобы построить график зависимости относительной погрешности ответа от возмущения.
	\begin{center}
		\includegraphics{Outrage}
	\end{center}
	Заметим, что погрешность увеличивается вместе с ростом возмущения, что логично, так как матрица A и вектор корней X остаётся тем же. При этом при минимальном возмущении порядок точности падает примерно в два раза по сравнению с точностью решения без возмущения правой части (\(10^{-7} \text{против} 10^{-15}\)).
	\section{Влияние размерности матрицы}
	Изучим, как меняется время выполнения подсчётов по методу прогонки в зависимости от размерности матрицы. Будем менять её, как описывалось в начале Главы 5. График будем строить так же с помощью пакета MATLAB.
	\begin{center}
		\includegraphics{Time}
	\end{center}
	Справедливо было ожидать, что при увеличении размерности матрицы, время выполнения тоже увеличивается. При этом можно заметить, что это возрастание не линейное, а экспоненциальное (\(\alpha \cdot n^3\)).
	\chapter{Общие выводы}
	В этой работе мы изучили прямые методы решения СЛАУ, в частности метод прогонки. Сравнили результаты этого метода с точным ответом, узнали, как на его выполнение влияют число обусловленности матрицы, возмущения свободных коэффициентов, а также размерность матрицы. Про метод прогонки можно сказать следующее: метод выполняется стабильно и достаточно быстро, требует всего \(O(n)\) арифметических операций. Внесенные изменения и осложнения не критично влияют на результат выполнения метода.
\end{document}