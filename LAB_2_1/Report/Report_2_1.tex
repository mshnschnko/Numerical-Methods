\documentclass[a4paper,14pt]{report}

\usepackage{cmap}
\usepackage[T2A]{fontenc}
\usepackage[utf8]{inputenc}
\usepackage[english,russian]{babel}
\usepackage{setspace}
\usepackage{geometry}
\geometry{top=2cm}
\geometry{bottom=2cm} % отступ снизу
\geometry{left=2cm} % отступ справа
\geometry{right=2cm} % отступ слева
\usepackage{graphicx}
\usepackage{amsmath}
\graphicspath{{./images/}}
\makeatletter %%%%% <---- Starting chapter without a pagebreak
\renewcommand\chapter{\par%
	\thispagestyle{plain}% \global\@topnum\z@
	\@afterindentfalse \secdef\@chapter\@schapter}
\makeatother

\begin{document}
	
	\begin{center}
		\hfill \break
		\begin{center}
			\huge{Санкт-Петербургский политехнический университет\\
				Высшая школа прикладной математики\\
				и вычислительной физики, ФизМех}
		\end{center}
		\hfill \break
		\hfill \break
		\hfill \break
		\hfill \break
		\hfill \break
		
		\huge{Направление подготовки\\
			<<Прикладная математика и информатика>>}\\
		\hfill \break
		\hfill \break
		\hfill \break
		\hfill \break
		\hfill \break
		\hfill \break
		\fontsize{14pt}{14pt}\selectfont
		Отчет по лабораторной работе №1\\
		<<Интерполяционные полиномы приближения табличных функций.\\
		Полином в форме Лагранжа>>\\
		\begin{center}
			Дисциплина: <<Численные методы>>
		\end{center}
		\hfill \break
		\hfill \break
		\hfill \break
		\hfill \break
		\hfill \break
	\end{center}
	\hfill \break
	\hfill \break
	\fontsize{12pt}{12pt}\selectfont 
	\begin{tabular}{cccc}
		\hspace{1cm}Выполнил студент гр. 5030102/00003 & {\hspace{3cm}} & & Анищенко М.Д. \\\\
		\hspace{-3cm}Преподаватель: &{\hspace{1cm}}& & {\hspace{1cm}} Курц В.В. \\\\
	\end{tabular}\\
	\hfill \break
	\hfill \break
	\hfill \break
	\hfill \break
	\hfill \break
	\hfill \break
	\hfill \break
	\hfill \break
	\begin{center} Санкт-Петербург 2022 \end{center}
	\fontsize{14pt}{14pt}\selectfont 
	\thispagestyle{empty}
	
	\newpage
	\tableofcontents{}
	\clearpage
	\chapter{Формулировка задачи и её формализация}
	\section{Формализация}
	Пусть известны пары \((x_i,y_i), i=0,1,\ldots,n\), а также \(x^h:=\{x_i\}^n_{i=0} \text{ --- сетка, } y^h:=\{y_i\}^n_{i=0}\) --- сеточная функция. Необходимо построить \(\phi(x)\), которая удовлетворяет условию близости: \(\phi(x)\approx(x^h, y^h)\),
	и \(\phi(x)\in C^{(k)} ([a,b])\text{, где } [a,b]\) --- отрезок, содержащий все \(x_i\).
	\section{Формулировка}
	В данном варианте требуется построить интерполяционный полином в форме Лагранжа, используя равномерную и Чебышевскую сетки, исследовать влияние количества узлов и их расположения на сходимость интерполяционного процесса.\\
	Исследования будем проводить с помощью функции \(f(x)=x^2\cos(2x)+1\) на отрезке \([-2,2]\).
	\chapter{Алгоритм метода и условия его применимости}
	\section{Алгоритм построения сеток}
	\begin{enumerate}
		\item Равномерная сетка: \(x_i=x_0+ih\).
		\item Чебышевская сетка: \(x_i=\cos(\frac{\pi(2i+1)}{2n+2}), i=0,1,\ldots,n\).
	\end{enumerate}
	\section{Алгоритм метода}
	Построим интерполяционный полином \(P_n(x):P_n(x_i)=y_i\).\\
	Будем искать \(P_n(x)\) в виде
	\[L_n(x)=\sum\limits_{i=0}^n y_i \Phi_i(x),\]
	где \(\Phi_i(x_j)=\delta_{ij}\).\\
	\(\Phi_i(x)=\alpha_i \prod\limits_{\substack{k=0\\k\neq i}}^n (x-x_k)\) --- \(i\)-ый базисный полином Лагранжа.\\
	\[\Phi_i(x_i)=1=\alpha_i \prod\limits_{\substack{k=0\\k\neq i}}^n (x_i-x_k)\Rightarrow\alpha_i=\ldots\Rightarrow\Phi_i(x)=\prod\limits_{\substack{k=0\\k\neq i}}^n \frac{(x-x_k)}{(x_i-x_k)}.\]\\
	Тогда интерполяционный полином в форме Лагранжа имеет вид:
	\[L_n(x)=\sum\limits_{i=0}^n y_i \prod\limits_{\substack{k=0\\k\neq i}}^n \frac{x-x_k}{x_i-x_k}.\]
	Также существует формула специально для равномерной сетки:
	\[x_i=x_0+ih, i=0,\ldots,n\]
	\[x=x_0+th, t\in[0,n]\]
	\[L_n(x_0+th)=\sum\limits_{i=0}^n y_i\frac{(t-0)(t-1)\ldots(t-i+1)(t-i-1)\ldots(t-n)}{(i-0)(i-1)\ldots 1(-1)\ldots(i-n)}\frac{t-i}{t-i}=\]
	\[=\sum\limits_{i=0}^n y_i\frac{(-1)^{n-i}\omega(t)}{i!(n-i)!(t-i)}\text{, где }\omega(t)=\prod\limits_{k=0}^n(t-k)\].
	\section{Условия применимости}
	Чтобы избежать деления на ноль, необходимо поставить условие для общей формулы \(\forall k \neq i: x_i \neq x_k\). Кроме того, функция \(f(x)\) должна быть непрерывной на отрезке \([a,b]\).
	\newpage
	\chapter{Анализ задачи}
	Задана табличная функция \((x_i,y_i), i=0,1,...,n\), потребуем выполнения условия интерполяции \(\phi(x_i)=y_i\), что можно записать в виде СЛАУ. Откуда следует, что интерполяционный полином существует и единственен, если степень полинома на единицу меньше количества узлом, и \(x_i\) попарно различны.\\
	Видим, что эти два условия будут выполнены при таком построении полинома, а также при соблюдении условий его применимости.
	\chapter{Выполение условий применимости}
	При использовании как равномерной, так и Чебышевской сетки точки \(x_i\) получаются попарно различными, что нам и требуется. Также для правильного построения полинома нужно, чтобы функция \(f(x)\) была непрерывной. Производная функции \(f(x)=x^2\cos2x+1\) имеет вид \(f'(x)=2x\cos(2x)-2x^2\sin2x\). Данная функция определена на всей вещественной прямой, а значит, исходная функция имеет производную на всем отрезке \([a,b] = [-2,2]\), следовательно, она непрерывна.
	\chapter{Тестовый пример}
	Построим интерполяционный полином в форме Лагранжа для нашей табличной функции \(y=x^2\cos(2x)+1\). Используем равномерную сетку на отрезке \([-2,2]\), для полинома 4-ой степени возьмём 5 точек.
	\begin{center}
		\begin{tabular}{|c|c|}
			\hline
			\(x_i\)& \(y_i\)\\\hline
			-2& -1.6146\\\hline
			-1& 0.5839\\\hline
			0& 1\\\hline
			1& 0.5839\\\hline
			2& -1.6146\\
			\hline
		\end{tabular}
	\end{center}
	Теперь по формуле для интерполяционного полинома Лагранжа вычисляем коэффициенты:
	\[L_n(x) = -1.6146\cdot \frac{(x-(-1))\cdot x\cdot(x-1)(x-2)}{(-2-(-1))(-2)(-2-1)(-2-2)}+\]
	\[+0.5839\cdot\frac{(x-(-2))\cdot x\cdot(x-1)(x-2)}{(-1-(-2))(-1)(-1-1)(-1-2)}+\]
	\[+1\cdot\frac{(x-(-2))(x-(-1))(x-1)(x-2)}{2\cdot1\cdot(-1)\cdot(-2)}+\]
	\[+0.5839\cdot\frac{(x-(-2))(x-(-1))\cdot x\cdot(x-2)}{(1-(-2))(1-(-1))\cdot1\cdot(1-2)}+\]
	\[+(-1.6146)\cdot\frac{(x-(-2))(x-(-1))\cdot x\cdot(x-1)}{(2-(-2))(2-(-1))\cdot2\cdot(2-1)}=\]
	\[=-0.079183x^4-0.336917x^2+1\]
	Мы получаем, что \(-0.079183x^4-0.336917x^2+1\approx x^2\cos(2x)+1\). Рассчитаем ошибку в неузловой точке. Например, возьмем \(x=0.5\) (середина отрезка между двумя узлами). Посчитаем значения начальной функции и интерполяционного полинома в этой точке. \(f(0.5)=1.13508, L_4(0.5)=0.910822\). Тогда ошибка равна \(0.224258\). При этом \(L_4(1)=0.5839\), что равняется значению \(f(1)\), как и должно быть. Как и ожидалось, точность интерполяции невелика, так как мы взяли всего пять узлов, соответственно полином имеет степень не выше четвертой. Условие интерполяции \(\phi(x_i)=y_i\) при этом выполняется.
	\begin{center}
		\includegraphics[scale=0.5]{test}
	\end{center}
	Теперь рассмотрим Чебышевскую сетку для того же количества узлов.
		\begin{center}
		\begin{tabular}{|c|c|}
			\hline
			\(x_i\)& \(y_i\)\\\hline
			1.90211& -1.85237\\\hline
			1.17557& 0.027755\\\hline
			0& 1\\\hline
			-1.17557& 0.027755\\\hline
			-1.90211& -1.85237\\
			\hline
		\end{tabular}
	\end{center}
	Применим ту же формулу для построения полинома.
	%\[L_n(x)=-1.85237\cdot\frac{(x-1.17557)\cdot x\cdot(x+1.17557)(x-1.90211))}{(1.90211-1.17557)\cdot1.90211\cdot(1.90211+1.17557)(1.90211+1.90211)}+\]
	%\[+0.027755\cdot\frac{(x-1.90211)\cdot x\cdot(x+1.17557)(x+1.90211))}{(1.17557-1.90211)\cdot1.17557\cdot(1.17557+1.17557)(1.17557+1.90211)}+\]
	%\[+1\cdot\frac{(x-1.90211)(x-1.17557)(x+1.90211)(x+1.17557)}{(-1.90211)(-1.17557)\cdot1.17557\cdot(1.90211)+}\]
	%\[+0.027755\cdot\frac{(x-1.90211)(x-1.17557)\cdot x\cdot(x-1.90211))}{(1.17557-1.90211)\cdot1.17557\cdot(1.17557-(-1.17557))(1.17557-(-1.90211))}+\]
	\[L_n(x)=-0,03794826041x^4-0,65108053395x^2+1\]
	В точке \(x=0.5\) получаем значение \(L_n(0.5)=0.8349, f(x)=1.1351\). Следовательно, ошибка равна \(0.3002\). В данном случае ошибка чуть больше, чем у равномерной сетки.
	\begin{center}
		\includegraphics[scale=0.5]{testCheb}
	\end{center}
	\chapter{Контрольные тесты}
	Для исследования выберем функцию \(f(x)=x^2\cos(2x)+1\). Применим для нахождения интерполяционного полинома разные сетки: равномерную и Чебышевскую. Изучим влияние количества узлов и их расположения а сходимость интерполяционного процесса. Для этого выберем отрезок \([-2, 2]\), количество узлов будем изменять от 4 до 8 с шагом в 2 для статичных графиков и построим зависимость ошибки интерполяции для каждой сетки при разном количестве узлов. Для динамичной картинки и влиянии большого количества узлов примем их значение от 6 до 61 с шагом 5.
	\newpage
	\chapter{Модульная структура программы}
	\begin{itemize}
		\item double f(double x) --- функция, вычисляющая значение \(f(x)\).
		\item double LagrangePolynom (const double* nodes, const double* y, double x, int nodesCount) --- функция, вычисляющая значения полинома Лагранжа. Принимает на вход набор пар \((x_i, y_i\), точку \(x\) для вычисления значения и количество узлов.
		\item double LagrangeForUniform(const double* y, double x, int nodesCount, double t) --- функция, аналогичная предыдущей, но в ней значения полинома вычисляются по специальной формуле для равноотстающих точек.
		\item void FillFile(FILE* file, double a, double b, const double* nodes, const double* y, int nodesCount) --- функция для заполнения файла значениями полинома для будущего построения графика функции и её исследования.
	\end{itemize}
	\newpage
	\chapter{Численный анализ}
	\section{Графики функций}
	Построим графики заданной функции и полученных полиномов.
	\begin{center}
		\includegraphics[scale=0.55]{UniformFunc1}
		\includegraphics[scale=0.55]{ChebFunc1}
		\includegraphics[scale=0.55]{UniformSpecFunc}
	\end{center}
	Мы можем видеть, что с увеличением числа узлов полиномы почти полностью сливаются с заданной функцией, то есть интерполяционный процесс действительно сходится.
	\newpage
	\section{Ошибка интерполяции}
	Посмотрим, какую ошибку интерполяции мы получаем для каждой сетки при разном количестве узлов.
	\begin{center}
		\includegraphics[scale=0.55]{UniError}
		\includegraphics[scale=0.55]{UniSpecError}
		\includegraphics[scale=0.65]{ChebError}
	\end{center}
	Для равномерной сетки наблюдается рост ошибки интерполяции ближе к краям отрезка. Для Чебышевской же сетки наоборот, наибольшую ошибку мы видим в середине отрезка.
	\newpage
	\section{Ошибка интерполяции при большом числе узлов}
	Теперь будем менять количество узлов от 6 до 61 и смотреть, как ведет себя максимальная ошибка интерполяции.
	\begin{center}
		\includegraphics[scale=0.85]{ErrorDependence}
	\end{center}
	Характерной чертой равномерной сетки является возрастание ошибки интерполяции, начиная с некоторого количества узлов. В данном случае с 26. Заметно, что ошибка значений полинома, найденных по специальной формуле для равноотстоящих точек, почти такая же, как и для общей формулы. Только при достаточно большом количестве узлов значения становятся чуть точнее.
	\chapter{Общие выводы}
	Исследования интерполяционного полинома в форме Лагранжа для разных сеток показали, что выбор сетки зависит от конкретных случаев и ситуаций. В случае, когда число узлов становится достаточно большим, использовать равномерную сетку уже нецелесообразно, поскольку она дает большую погрешность (вероятно, точка этого преломления зависит от функции). При этом для небольшого количества ее удобно использовать, потому что вычисление точек \(x_i\) проще в данном случае. Также важно учитывать, что наибольшая ошибка для равномерной и Чебышевской сеток наблюдается в противоположных местах. Для первой --- на концах отрезка, а для второй --- ближе к середине. Однако для обоих случаев можно сказать, что интерполяционный полином Лагранжа достаточно близок к заданной функции (при подходящем количестве узлов), а значит пригоден для использования.
\end{document}