\documentclass[a4paper,14pt]{report}

\usepackage{cmap}
\usepackage[T2A]{fontenc}
\usepackage[utf8]{inputenc}
\usepackage[english,russian]{babel}
\usepackage{setspace}
\usepackage{geometry}
\geometry{top=2cm}
\geometry{bottom=2cm} % отступ снизу
\geometry{left=2cm} % отступ справа
\geometry{right=2cm} % отступ слева
\usepackage{graphicx}
\usepackage{amsmath}
\usepackage{amsfonts}
\graphicspath{{./images/}}
\makeatletter %%%%% <---- Starting chapter without a pagebreak
\renewcommand\chapter{\par%
	\thispagestyle{plain}% \global\@topnum\z@
	\@afterindentfalse \secdef\@chapter\@schapter}
\makeatother

\begin{document}
	
	\begin{center}
		\hfill \break
		\begin{center}
			\huge{Санкт-Петербургский политехнический университет\\
				Высшая школа прикладной математики\\
				и вычислительной физики, ФизМех}
		\end{center}
		\hfill \break
		\hfill \break
		\hfill \break
		\hfill \break
		\hfill \break
		
		\huge{Направление подготовки\\
			<<Прикладная математика и информатика>>}\\
		\hfill \break
		\hfill \break
		\hfill \break
		\hfill \break
		\hfill \break
		\hfill \break
		\fontsize{14pt}{14pt}\selectfont
		Отчет по лабораторной работе №2\\
		<<Приближение табличных функций с использованием квадратичного сплайна с производной на левом конце.>>\\
		\begin{center}
			Дисциплина: <<Численные методы>>
		\end{center}
		\hfill \break
		\hfill \break
		\hfill \break
		\hfill \break
		\hfill \break
	\end{center}
	\hfill \break
	\hfill \break
	\fontsize{12pt}{12pt}\selectfont 
	\begin{tabular}{cccc}
		\hspace{1cm}Выполнил студент гр. 5030102/00003 & {\hspace{3cm}} & & Анищенко М.Д. \\\\
		\hspace{-3cm}Преподаватель: &{\hspace{1cm}}& & {\hspace{1cm}} Курц В.В. \\\\
	\end{tabular}\\
	\hfill \break
	\hfill \break
	\hfill \break
	\hfill \break
	\hfill \break
	\hfill \break
	\hfill \break
	\hfill \break
	\begin{center} Санкт-Петербург 2022 \end{center}
	\fontsize{14pt}{14pt}\selectfont 
	\thispagestyle{empty}
	
	\newpage
	\tableofcontents{}
	\clearpage
	\chapter{Формулировка задачи и её формализация}
	\section{Формализация}
	Пусть даны \(x_0,\ldots,x_n\) - \(n+1\) различных узлов на отрезке \([a,b]\), где \(a=x_0<x_1<\ldots<x_n<b\). Функция \(S^\nu_k\) на отрезке \([a,b]\) называется сплайном степени \(k\) отнисительно узлов \(x_i\), если
	\[S^\nu_k|_{[x_i,x_{i+1}]}\in\mathbb{P}_k, i =0,1,\ldots,n-1\]
	\[S^\nu_k \in C^{k-\nu}([a,b])\]
	Пусть известны пары \((x_i,y_i),i=0,1,\ldots,n\) и значение производной \(d_0\) заданной функции на левом конце отрезка \([a,b]\)
	\section{Формулировка}
	Необходимо построить кусочно-заданную функцию \(S^1_2\), являющуюся квадратичным сплайном такую, что
	\[S^1_2|_{[x_i,x_{i+1}]}=g_i(x)=a_ix^2+b_ix+c_i, i=1,\ldots,n\]
	А также исследовать влияние количества узлов на сходимость процесса, сравнить с интерполяционным полиномом Лагранжа и дополнительно исследовать влияние точности задания дополнительных условий на точность результата. Исследования будем проводить на функции \(f(x)=x^2\cos(2x)+1\) на отрезке \([-2,2]\).
	\newpage
	\chapter{Алгоритм метода и условия его применимости}
	\section{Условия применимости}
	\begin{enumerate}
		\item Узлы \(x_i\) попарно различны
		\item Функция \(f(x)\) и её производная \(f'(x)\) непрерывны на отрезке \([a,b]\)
	\end{enumerate}
	\section{Алгоритм метода}
	На каждом отрезке \([x_i,x_{i+1}]\) необходимо найти полином \(g_i(x)=a_ix^2+b_ix+c_i\), удовлетворяющий условиям
	$$
	\begin{cases}
		g_i(x_{i-1})=y_{i-1}\\
		g_i(x_i)=y_i\\
	\end{cases}
	i=1,\ldots,n \text{ и } g'_i(x_i)=g'_{i+1}(x_i), i=1,\ldots,n-1
	$$
	Объединяя эти условия и принимая во внимание то, что значение производной \(d_0\) на левом конце у нас известно, можем составить следующую систему условий для первого отрезка, с помощью которой мы однозначно разрешим значения \(a_1,b_1,c_1\) и дальше найдем полиномы для остальных отрезков.
	$$
	\begin{cases}
		g_1(x_0)=y_0\\
		g_1(x_1)=y_1\\
		g'_1(x_0)=d_0\\
	\end{cases}
	\Rightarrow a_1,b_1,c_1 \Rightarrow g'_1(x_1) \Rightarrow 
	\begin{cases}
		g_2(x_1)=y_1\\
		g_2(x_2)=y_2\\
		g'_2(x_1)=g'_1(x_1)\\
	\end{cases}
	\Rightarrow \ldots
	$$
	\chapter{Анализ задачи}
	Задана табличная функция \((x_i,y_i), i=0,1,...,n\), потребуем выполнения условия интерполяции на каждом подотрезке \([x_i,x_{i+1}]: \phi(x_i)=y_i\), что можно записать в виде СЛАУ. Откуда следует, что интерполяционный полином существует и единственен для каждого указанного подотрезка.
	\newpage
	\chapter{Выполение условий применимости}
	При использовании как равномерной сетки точки \(x_i\) получаются попарно различными, что нам и требуется. Также для правильного построения полинома нужно, чтобы функция \(f(x)\) была непрерывной. Производная функции \(f(x)=x^2\cos2x+1\) имеет вид \(f'(x)=2x\cos(2x)-2x^2\sin2x\). Данная функция определена на всей вещественной прямой, а значит, исходная функция имеет производную на всем отрезке \([a,b] = [-2,2]\), следовательно, она непрерывна.
	\chapter{Тестовый пример}
	Построим квадратичный сплайн для табличной функции \(y=x^2\cos(2x)+1\). Используем равномерную сетку на отрезке \([-2,2]\), возьмём 6 точек.
	\begin{center}
		\begin{tabular}{|c|c|c|}
			\hline
			\(i\)& \(x_i\)& \(y_i\)\\\hline
			0& -2& -1.6146\\\hline
			1& -1.2& -0.0618\\\hline
			2& -0.4& 1.1115\\\hline
			3& 0.4& 1.1115\\\hline
			4& 1.2& -0.0618\\\hline
			5& 2& -1.6146\\
			\hline
		\end{tabular}
	\end{center}
	Теперь по вышеописанным формулам вычисляем полиномы. Преобразуем СЛАУ, выражая коэффициенты \(a_i,b_i,c_i\).
	$$
	\begin{cases}
		a_1 = \frac{y_1-y_0-d_0(x_1-x_0)}{(x_1-x_0)^2}\\
		b_1 = d_0-2a_1x_0\\
		c_1 = y_0-a_1x_0^2-b_1x_0\\
	\end{cases}
	$$
	Аналогично для следующих шагов получаем коэффициенты с новыми индексами (0 сменяется на \(i\), 1 на \(i+1\) в общем виде), значения производных мы находим после нахождения коэффициентов: \(g'(x)=2ax+b\).
	\newpage
	$$
	\begin{cases}
		a_1=\frac{-0.0618+1.6146+3.43985(-1.2-(-2))}{(-1.2-(-2))^2}\\
		b_1=-3.43985+2\cdot6.72606\cdot2\\
		c_1=-1.6146+6.72606\cdot4+23.46439\cdot2\\
		g'_1(x_1)=2\cdot6.72606\cdot(-1.2)+23.46439\\
	\end{cases}
	\begin{cases}
		a_1=6.72606\\
		b_1=23.46439\\
		c_1=18.40954\\
		g'_1(x_1)=7.321846\\
	\end{cases}
	$$
	$$
	\begin{cases}
		a_2=\frac{1.1115+0.0618-7.321846(-0.4-(-1.2))}{(-0.4-(-1.2))^2}\\
		b_2=7.321846+2\cdot7.31903\cdot(-1.2)\\
		c_2=-0.0618+7.31903\cdot1.2^2-10.243826\\
		g'_2(x_2)=2\cdot7.31903\cdot0.4-10.243826\\
	\end{cases}
	\begin{cases}
		a_2=-7.31903\\
		b_2=-10.243826\\
		c_2=-1.814988\\
		g'_2(x_2)=-4.388602\\
	\end{cases}
	$$
	$$
	\begin{cases}
		a_3=\frac{1.1115-1.1115+4.988602(0.4-(-0.4))}{(0.4-(-0.4))^2}\\
		b_3=-4.388602-2\cdot5.48575\cdot(-0.4)\\
		c_3=1.1115-5.48575\cdot(-0.4)^2+0.000002\cdot(-0.4)\\
		g'_3(x_3)=2\cdot5.48575\cdot0.4-0.233779\\
	\end{cases}
	\begin{cases}
		a_3=5.48575\\
		b_3=-0.000002\\
		c_3=-0.233779\\
		g'_3(x_3)=4.154821\\
	\end{cases}
	$$
	$$
	\begin{cases}
		a_4=\frac{-0.0618-1.1115-4.154821(1.2-0.4)}{(1.2-0.4)^2}\\
		b_4=4.154821+2\cdot7.31877\cdot0.4\\
		c_4=1.1115+7.31877\cdot0.4^2-10.243826\cdot0.4\\
		g'_4(x_4)=2\cdot(-7.31903)\cdot1.2+10.243826\\
	\end{cases}
	\begin{cases}
		a_4=-7.31903\\
		b_4=10.243826\\
		c_4=-1.814988\\
		g'_4(x_4)=-7.321846\\
	\end{cases}
	$$
	$$
	\begin{cases}
		a_5=\frac{-1.6146+0.0618+7.321846(2-1.2)}{(2-1.2)^2}\\
		b_5=-7.321846+2\cdot6.72606\cdot1.2\\
		c_5=-0.0618-6.72606\cdot1.2^2+23.46439\cdot1.2\\
		g'_5(x_5)=2\cdot6.72606\cdot2-23.46439\\
	\end{cases}
	\begin{cases}
		a_5=6.72606\\
		b_5=-23.46439\\
		c_5=18.40954\\
		g'_5(x_5)=3.43985\\
	\end{cases}
	$$
	Теперь построим график по полученным пяти квадратным полиномам.
	\begin{center}
		\includegraphics[scale=0.7]{teset1}
	\end{center}
	\newpage
	Видно, что график, построенный с помощью квадратичного сплайна, приближается к графику нашей функции. Вычислим ошибку в точке \(x=0.6\). Получаем \(g_4(0.6) = 1.6964\), значение функции в той же точке \(f(0.6) = 1.1304\). Тогда ошибка вычисления равна \(\delta=1.6964-1.1304=0.566\). Ошибка весьма велика, однако видно, что это одно из наибольших значений погрешности на отрезке. Значит, метод сходится.
	\chapter{Контрольные тесты}
	Для исследования выберем функцию \(f(x)=x^2\cos(2x)+1\). Применим для нахождения интерполяционного полинома равномерную сетку. Изучим влияние количества узлов и их расположения а сходимость интерполяционного процесса. Для этого выберем отрезок \([-2, 2]\), количество узлов будем изменять от 4 до 8 с шагом в 2 для статичных графиков и построим зависимость ошибки интерполяции при разном количестве узлов. Для динамичной картинки и влиянии большого количества узлов примем их значение от 6 до 181 с шагом 5. Дополнительно изучим влияние погрешности в начальных даных. В данно случае это значение производной на левом конце.
	\chapter{Модульная структура программы}
	\begin{itemize}
		\item double f(double x) --- функция, вычисляющая значение \(f(x)\).
		\item double derf(double x) --- функция, вычисляющая значение производной \(f'(x)\).
		\item void QuadraticSplineMethod(const double x[], const double y[], polynom g[], int n, const double delta) --- функция, вычисляющая значения полиномов. Принимает на вход набор пар \((x_i, y_i\), массив полиномов \(g_i\) для вычисления коэффициентов, количество узлов и возмущение (для последнего пункта исследования).
		\item void FillFile(FILE* file, double a, double b, const double* nodes, const double* y, int n, const double delta) --- функция для заполнения файла значениями полиномов для будущего построения графика функции и её исследования.
	\end{itemize}
	\chapter{Численный анализ}
	\section{Графики функций}
	Построим графики заданной функции и полученных полиномов.
	\begin{center}
		\includegraphics[scale=0.8]{func}
	\end{center}
	Мы видим, что с увеличением количества полиномов, точность увеличивается и график для 8 узлов почти сливается с графиком функции. Следовательно, сходимость достигается.
	\newpage
	\section{Ошибка интерполяции}
	Построим график фактической ошибки на всём отрезке.
	\begin{center}
		\includegraphics[scale=0.55]{SplineError}
		\includegraphics[scale=0.55]{LagrangeError}
	\end{center}
	Для равномерной сетки в полиноме Лагранжа наблюдается рост ошибки интерполяции ближе к краям отрезка. В случае квадратичного сплайна картина обратная - наибольшая ошибка достигается в центре отрезка.
	\section{Ошибка интерполяции при большом числе узлов}
	Теперь будем менять количество узлов от 6 до 181 и смотреть, как ведет себя максимальная ошибка интерполяции в зависимости от расстояния между двумя узлами. Сравним с ошибкой полинома Лагранжа. Для удобства рассмотрим два разных способа построение (логарифмическую и обычную оси Ox).
	\begin{center}
		\includegraphics[scale=0.55]{dependence1}
		\includegraphics[scale=0.55]{dependence2}
	\end{center}
	Рассмотрев точки на графике, заметим, что зависимость примерно равна \(O(h^3)\). Кроме того, квадратичный сплайн лишен главного минуса метода Лагранжа, а именно быстрого возрастания ошибки после некоторого числа узлов. Но вместо этого мы получаем заметно меньшую точность, при количестве узлов около 180 мы получаем значения функций с точностью до \(10^{-6}\), в то время как полином Лагранжа имеет максимальную точность порядка \(10^{-11}\).
	\section{Зависимость возмущения начальных данных}
	Внесём возмущение в значение производной на левом конце. Возмущение будет принимать значения от \(10^{-5}\%\) до \(100000\%\).
	\begin{center}
		\includegraphics[scale=0.55]{pert1}
		\includegraphics[scale=0.55]{pert2}
	\end{center}
	\begin{center}
		\includegraphics[scale=0.55]{pert3}
		\includegraphics[scale=0.55]{pert5}
	\end{center}
	По полученным результатам отметим, что видимое изменение достигается при возмущении около \(10\%\)), при \(100\%\) видно, что графики имеют общие формы, но ошибка уже заметна. При возмущении около \(100000\%\) в графике не проглядывается ничего общего с графиком нашей функции, а ошибка достигает значения около 172. Делаем вывод, что метод устойчив к возмущению входним данных, если это возмущение составляет примерно \(10\%\) и меньше.
	\newpage
	\chapter{Общие выводы}
	По результатам наших исследований мы видим, что метод квадратичного сплайна более удобный и надежный, чем метод полинома Лагранжа, однако имеем меньшую точность, что необходимо учитывать при выборе метода для конкретной задачи. Если важна точность, а число узлов невелико, то стоит предпочесть полином Лагранжа. Если же нужно использоать большое количество узлов, но можно принебречь точностью в разумных пределах, то следует выбрать квадратичный сплайн.
\end{document}