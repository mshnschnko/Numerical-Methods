\documentclass[a4paper,14pt]{report}

\usepackage{cmap}
\usepackage[T2A]{fontenc}
\usepackage[utf8]{inputenc}
\usepackage[english,russian]{babel}
\usepackage{setspace}
\usepackage{geometry}
\geometry{top=2cm}
\geometry{bottom=2cm} % отступ снизу
\geometry{left=2cm} % отступ справа
\geometry{right=2cm} % отступ слева
\usepackage{graphicx}
\usepackage{amsmath}
\usepackage{amsfonts}
\graphicspath{{./images/}}
\makeatletter %%%%% <---- Starting chapter without a pagebreak
\renewcommand\chapter{\par%
	\thispagestyle{plain}% \global\@topnum\z@
	\@afterindentfalse \secdef\@chapter\@schapter}
\makeatother

\begin{document}
	
	\begin{center}
		\hfill \break
		\begin{center}
			\huge{Санкт-Петербургский политехнический университет\\
				Высшая школа прикладной математики\\
				и вычислительной физики, ФизМех}
		\end{center}
		\hfill \break
		\hfill \break
		\hfill \break
		\hfill \break
		\hfill \break
		
		\huge{Направление подготовки\\
			<<Прикладная математика и информатика>>}\\
		\hfill \break
		\hfill \break
		\hfill \break
		\hfill \break
		\hfill \break
		\hfill \break
		\fontsize{14pt}{14pt}\selectfont
		Отчет по лабораторной работе №5\\
		<<Численное интегрирование ОДУ одношаговыми методами>>\\
		\begin{center}
			Дисциплина: <<Численные методы>>
		\end{center}
		\hfill \break
		\hfill \break
		\hfill \break
		\hfill \break
		\hfill \break
	\end{center}
	\hfill \break
	\hfill \break
	\fontsize{12pt}{12pt}\selectfont 
	\begin{tabular}{cccc}
		\hspace{1cm}Выполнил студент гр. 5030102/00003 & {\hspace{3cm}} & & Анищенко М.Д. \\\\
		\hspace{-3cm}Преподаватель: &{\hspace{1cm}}& & {\hspace{1cm}} Курц В.В. \\\\
	\end{tabular}\\
	\hfill \break
	\hfill \break
	\hfill \break
	\hfill \break
	\hfill \break
	\hfill \break
	\hfill \break
	\hfill \break
	\begin{center} Санкт-Петербург 2022 \end{center}
	\fontsize{14pt}{14pt}\selectfont 
	\thispagestyle{empty}
	
	\newpage
	\tableofcontents{}
	\clearpage
	\chapter{Формулировка задачи и её формализация}
	\section{Формулировка задачи}
	Пусть дано дифференциальное уравнение 1-го порядка
	\[F(x,y,y')=0\text{,}\]
	где \(y(x)\) --- неизвестная функция. Поставлена задача Коши:
	$$
	\begin{cases}
		y'=f(x,y), x \in [a,b]\\
		y(a) = y_0
	\end{cases}
	$$
	Необходимо найти приближенное решение этой задачи.
	\section{Формализация задачи}
	В данном варианте необходимо решить следующую задачу Коши методом Эйлера:
	$$
	\begin{cases}
		(2x+1)y'=4x+2y, x \in [0,4]\\
		y(0) = 1
	\end{cases}
	$$
	Из условия известно точное решение, с которым мы будем сравнивать решение, полученное методом Эйлера:
	\[y=(2x+1)\ln|2x+1|+1\]
	\chapter{Алгоритм метода и условия его применимости}
	\section{Алгоритм метода}
	\begin{enumerate}
		\item Строим равномерную сетку: \[x_k=a+kh, k=0,\ldots,n; h=\frac{b-a}{n}\]
		\[x, x+h \in [a,b]\]
		\item Разложим в ряд Тейлора:
		\[y(x+h)=y(x)+hy'(x)+\mathcal{O}(h^2)\]
		Пусть \(x=x_k\Rightarrow y(x_k+h)=y(x_k)+hf(x_k,y(x_k))+\mathcal{O}(h^2)\).\\
		\item Отбросим \(\mathcal{O}(h^2)\) и заменим \(y(x_k)\) на \(y_k\).
		\[y_{k+1}=y_k+hf(x_k,y_k)\]
	\end{enumerate}
	\section{Условия применимости}
	Необходимо, чтобы \(f(x,y)\) была определена на всем отрезке \([a,b]\).
	\chapter{Анализ задачи}
	Для корректного решения задачи, мы должны проверить выполнение условий применимости.
	\[y'=f(x,y)=\frac{4x+2y}{2x+1}\]
	\[2x+1\neq0\Rightarrow x\neq-\frac{1}{2}\]
	В данном варианте вычисления проводятся на отрезке \([0,4]\).\\
	\(-\frac{1}{2}\notin[0,4]\). Значит, условия применимости выполнены.
	\chapter{Тестовый пример}
	Тестовый пример выполним для нашей функции \(F(x,y,y')\). Сделаем два шага с длиной \(h=0.1\) и один шаг с длиной \(2h=0.2\), затем вычислим ошибку на каждом этапе и применим правило Рунге. 
	$$
	\begin{cases}
		y'=\frac{4x+2y}{2x+1}\\
		y(0)=1
	\end{cases}
	$$
	\begin{enumerate}
		\item Два шага длиной \(h=0.1\):
		\begin{enumerate}
			\item \[y_1=y_0+hf(x_0,y_0)=1+0.1\cdot\frac{4\cdot0+2\cdot1}{2\cdot0+1}=1+0.1\cdot2=1.2\]
			Фактическая ошибка равна \(|y(0.1)-y_1|=|1.21879-1.2|=0.01879\).
			\item \[y_2=y_1+hf(x_1,y_1)=1.2+0.1\cdot\frac{4\cdot0.1+2\cdot1.2}{2\cdot0.1+1}=1.2+0.1\cdot2.33=1.433\]
			Фактическая ошибка равна \(|y(0.2)-y_2|=|1.47106-1.433|=0.0380611\).
		\end{enumerate}
		\item Один шаг длиной \(2h=0.2\):
		\[y_1=y_0+hf(x_0,y_0)=1+2\cdot0.1\cdot\frac{4\cdot0+2\cdot1}{2\cdot0+1}=1+2\cdot0.1\cdot2=1.4\]
		Фактическая ошибка равна \(|y(0.2)-y_1|=|1.47106-1.4|=0.0710611\).
	\end{enumerate}
	Проверка по правилу Рунге:
	\[|y_{2h}-y_h|=|1.4-1.433|=0.033\]
	Мы видим, что при \(h=0.1\) и \(2h=0.2\) мы получили ответ с ошибкой порядка \(10^{-2}\) с точностью до константы.
	\newpage
	\chapter{Контрольные тесты}
	\begin{enumerate}
		\item Сравнить графики точного и численного решения. Для этого мы построим график, разбивая отрезок на 25 частей.
		\item Построить график ошибки.
		\item Исследовать зависимость локальной и глобальной погрешностей от \(h\).
		\item Проверить, достигается ли точность с применением правила Рунге.
		\item Исследовать влияние заданной точности на объем вычислений.
	\end{enumerate}
	\chapter{Модульная структура программы}
	\begin{itemize}
		\item double f(double x, double y) --- функция для вычисления значения \(y'\) в точке \(x\).
		\item double* CreateGrid(double a, double b, int n) --- функция для создания равномерной сетки.
		\item double* Euler(double a, double b, int n, double f\_a) --- функция, реализующая метод Эйлера на равномерной сетке, т.е. без правила Рунге, а с заданным числом отрезков \(n\).
		\item int Error(double y\_2h, double y\_h, double epsl) --- функция для вычисления ошибки между приближениями \(y_h\) и \(y_{2h}\) (правило Рунге). Если ошибка больше требуемой точности, то возвращается значение 1, иначе 0. Значение 1 означает, что нужно продолжать уменьшать \(h\).
		\item double StepEulerRunge(double x, double h, double y\_0, int \&k) --- функция для нахождения \(y_{k+1}\) и подсчета объема вычислений.
		\item double Euler\_Runge(double a, double b, double epsl, int\& n, double f\_a, int\& k) --- функция, реализующая метод Эйлера с приминением правила Рунге.
	\end{itemize}
	\chapter{Численный анализ}
	\section{Графики функций}
	\begin{center}
		\includegraphics[scale=0.7]{func}
	\end{center}
	Мы можем видеть, что уже при 26 узлах график функции, полученной из численного метода, достаточно близок к точному графику функции \(y(x)\).
	\section{График ошибки}
	\begin{center}
		\includegraphics[scale=0.7]{errorone}
	\end{center}
	Из данного графика мы видим, что при увеличении значения \(x\) ошибка увеличивается. Это объясняется тем, что в алгоритме метода мы отбросили \(\mathcal{O}(h^2)\). Соответственно, это отброшенное значение есть локальная ошибка. И так как каждое следующее значение \(y_{k+1}\) получается из \(y(_k)\), так же отбрасывая \(\mathcal{O}(h^2)\), ошибка накапливается. График соответствует этому.
	\section{Зависимость локальной и глобальной погрешностей от \(h\)}
	\begin{center}
		\includegraphics[scale=0.7]{error_wa_runge}
	\end{center}
	\newpage
	Из графика мы видим, что локальная погрешность зависит от \(h\) как \(\mathcal{O}(h^2)\), потому что именно таким значением мы пренебрегали при нахождении каждого следующего \(y\). При этом глобальная ошибка уменьшается, как и ожидалось, потому что большее количество узлов соответствует более точному результату (что также подтвердилось в тестовом примере). Однако разница между глобальной и локальной погрешностью увеличивается при уменьшении точности, поскольку количество таких пренебрежений \(\mathcal{O}(h^2)\) возрастает.
	\section{Достижение точности с применением правила Рунге}
	\begin{center}
		\includegraphics[scale=0.7]{runge_error}
	\end{center}
	Данный график показывает, что праивло Рунге действительно контролирует погрешность, вследствие чего достигается требуемая точность.
	\section{Объем вычислений}
	\begin{center}
		\includegraphics[scale=0.7]{amount}
	\end{center}
	Из графика видно, что объем вычислений растет, причем нелинейно. При первых увеличениях точности объем возрастает в 8 раз, затем в 16, потом и вовсе в 64. Именно поэтому в данных исследованиях точность была ограничена \(10^{-6}\), так как при более высоких порядках объем вычислений был слишком велик и требовал много времени.
	\chapter{Общие выводы}
	Исходя из результатов исследования, можно сказать, что метод Эйлера легко реализуем, но из-за малого порядка точности требует большого обьъема вычислений. Плюсом данного метода также являются "слабые" условия применимости. Однако, его не следует применять на практике для нахождения решения с высокой точностью, поскольку он неэффективен по сравнению с методами более высоких порядков, например, модифицированным методом Эйлера (2 порядок точности) или методами Рунге-Кутты (3 и 4 порядки точности).
\end{document}