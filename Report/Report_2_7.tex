\documentclass[a4paper,14pt]{report}

\usepackage{cmap}
\usepackage[T2A]{fontenc}
\usepackage[utf8]{inputenc}
\usepackage[english,russian]{babel}
\usepackage{setspace}
\usepackage{geometry}
\geometry{top=2cm}
\geometry{bottom=2cm} % отступ снизу
\geometry{left=2cm} % отступ справа
\geometry{right=2cm} % отступ слева
\usepackage{graphicx}
\usepackage{amsmath}
\usepackage{amsfonts}
\graphicspath{{./images/}}
\makeatletter %%%%% <---- Starting chapter without a pagebreak
\renewcommand\chapter{\par%
	\thispagestyle{plain}% \global\@topnum\z@
	\@afterindentfalse \secdef\@chapter\@schapter}
\makeatother

\begin{document}
	
	\begin{center}
		\hfill \break
		\begin{center}
			\huge{Санкт-Петербургский политехнический университет\\
				Высшая школа прикладной математики\\
				и вычислительной физики, ФизМех}
		\end{center}
		\hfill \break
		\hfill \break
		\hfill \break
		\hfill \break
		\hfill \break
		
		\huge{Направление подготовки\\
			<<Прикладная математика и информатика>>}\\
		\hfill \break
		\hfill \break
		\hfill \break
		\hfill \break
		\hfill \break
		\hfill \break
		\fontsize{14pt}{14pt}\selectfont
		Отчет по лабораторной работе №7\\
		<<Решения краевых задач>>\\
		\begin{center}
			Дисциплина: <<Численные методы>>
		\end{center}
		\hfill \break
		\hfill \break
		\hfill \break
		\hfill \break
		\hfill \break
	\end{center}
	\hfill \break
	\hfill \break
	\fontsize{12pt}{12pt}\selectfont 
	\begin{tabular}{cccc}
		\hspace{1cm}Выполнил студент гр. 5030102/00003 & {\hspace{3cm}} & & Анищенко М.Д. \\\\
		\hspace{-3cm}Преподаватель: &{\hspace{1cm}}& & {\hspace{1cm}} Курц В.В. \\\\
	\end{tabular}\\
	\hfill \break
	\hfill \break
	\hfill \break
	\hfill \break
	\hfill \break
	\hfill \break
	\hfill \break
	\hfill \break
	\begin{center} Санкт-Петербург 2022 \end{center}
	\fontsize{14pt}{14pt}\selectfont 
	\thispagestyle{empty}
	
	\newpage
	\tableofcontents{}
	\clearpage
	\chapter{Формулировка задачи и её формализация}
	\section{Формулировка задачи}
	Дано линейное ОДУ 2-го порядка с переменными коэффициентами.
	\begin{equation}
		\label{137}p(x)y''+q(x)y'+r(x)y=f(x),
	\end{equation}
	%\[\label{137}p(x)y''+q(x)y'+r(x)y=f(x),\]
	\[p(x), q(x), r(x), f(x) \in C([a,b]).\]
	Дифференциальный оператор \(L=p\frac{d^2}{dx^2}+q\frac{d}{dx}+r\). Тогда уравнение выше можно записать в виде \(L(y)=f\).
	Общий вид граничных условий:
	\begin{equation}
		\label{138}
		\begin{cases}
			\alpha_0 y(a)+\alpha_1y'(a)=A\\
			\beta_0 y(b)+\beta_1 y'(b)=B
		\end{cases}
	\end{equation}
	\[\alpha_0^2+\alpha_1^2 \neq 0, \beta_0^2+\beta_1^2\neq0.\]
	Краевая задача состоит из ~(\ref{137}) и ~(\ref{138}).
	\section{Формализация задачи}
	В данном варианте необходимо решить следующую краевую задачу методом конечных разностей. Причем \(\alpha_1\neq0\).
	\begin{equation}
		\label{my_task}
		\begin{cases}
			y''+4xy'+(4x^2+3)y=e^{-x^2}\\
			u(0)=1\\
			u(1)=\frac{1}{e}
		\end{cases}
	\end{equation}
	Точное решение данной задачи: \(e^{-x^2}\).
	\newpage
	\chapter{Алгоритм метода и условия его применимости}
	\section{Алгоритм метода}
	Заменим производные конечными разностями на концах отрезка:
	\begin{equation}
		\label{ends}
		\frac{dy(x_0)}{dx}=\frac{y_1-y_0}{h}, \frac{dy(x_k)}{dx}=\frac{y_k-y_{k-1}}{h}\\
	\end{equation}
	И в промежуточных точках \(x_i\):
	\begin{equation}
		\label{intermediate}
		\frac{dy}{dx} \approx \frac{\Delta y}{\Delta x} = \frac{y_{i+1}-y_{i-1}}{x_{i+1}-x_{i-1}} = \frac{y_{i+1}-y_{i-1}}{2h}
	\end{equation}
	Заменим вторую производную конечными разностями:
	\begin{equation}
		\label{second_dir}
		\frac{d^2y}{dx^2} = \frac{\Delta(\Delta y)}{\Delta(\Delta x)} = \frac{(y_{i+1}-y_i)-(y_i-y_{i-1})}{h^2} = \frac{y_{i+1} - 2y_i + y_{i-1}}{h^2}
	\end{equation}
	Теперь вместо ДУ будем решать СЛАУ:
	\begin{equation}
		\label{SLE}
		\begin{cases}
			\alpha_0 y_0 + \alpha_1 \frac{y_1-y_0}{h} = A\\
			p_i \frac{y_{i+1} - 2y_i + y_{i-1}}{h^2} + q_i \frac{y_{i+1}-y_{i-1}}{2h} + r_i y_i = f_i\\
			\beta_0 y_k + \beta_1 \frac{y_k-y_{k-1}}{h} = B
		\end{cases}
	\end{equation}
	Умножим первое и последнее уравнение на \(h\), остальные на \(h^2\). Запишем это в матричном виде и получим
	\begin{equation}
		\label{matrixEq}
		W\cdot Y = F,
	\end{equation}
	где \(Y\) --- вектор переменных, \(F\) --- вектор свободных членов (\(F_1=Ah, F_i = f_ih^2, F_{k+1}=Bh\)), \(W\) --- трехдиагональная матрица размерности\\\((k+1)\times(k+1)\).
	\begin{equation}
		\label{matrixW}
		W=
		\begin{pmatrix}
			\alpha_0h-\alpha_1& \alpha_1 & 0& 0& \cdots& 0& 0& 0\\
			1-\frac{q_1h}{2}& r_1h^2-2& 1+\frac{q_1h}{2}& 0& \cdots& 0& 0& 0\\
			0& 1-\frac{q_2h}{2}& r_2h^2-2& 1+\frac{q_2h}{2}& \cdots& 0& 0& 0\\
			0& 0& \cdots& \cdots& \cdots& \cdots& \cdots& 0\\
			0& 0& 0& 0& \cdots& 1-\frac{q_kh}{2}& r_kh^2-2& 1+\frac{qkh}{2}\\
			0& 0& 0& 0& \cdots& 0& -\beta_1& \beta_0h + \beta_1
		\end{pmatrix}
	\end{equation}
	Решая эту матрицу методом прогонки, мы получим искомый вектор \(Y\).
	\section{Условия применимости}
	Из теоремы известно, что если \(\forall x\in [a,b]\)
	\begin{equation}
		\label{theory}
		\begin{cases}
			p(x)\geq0\\
			p(x)\geq \frac{h}{2}|q(x)|\\
			r(x)\leq 0
		\end{cases}
	\end{equation}
	то СЛАУ (\ref{my_task}) имеет единственное решение. Это условие является достаточным и указывает на диагональное преобладание.
	\chapter{Анализ задачи}
	В данном варианте мы имеем \(p(x)=1, q(x)=4x, r(x)=(4x^2+3)\). Выберем следующие коэффициенты: \(\alpha_0=2, \alpha_1=-2.5, \beta_0=3, \beta_1=0\). Тогда при наших условиях краевой задачи получаем \(A=2, B=\frac{3}{e}\). Установим диагональное преобладание в первой и последней строках:
	\[\alpha_0h-\alpha_1=2h-(-2.5)=2h+2.5\]
	Так как \(h\) всегда положитеьно, то это число больше \(2.5\). При этом \(|\alpha_1|=2.5\). Значит, диагональное преобладание присутствует.
	\[\beta_0h+\beta_1=3h+0=3h\]
	Так как \(\beta_1=0\), а \(h\) всегда положительно, то диагональное преобладание здесь также присутствует.
	\chapter{Тестовый пример}
	Протестируем метод, разбив данный отрезок \([0,1]\) на 5 частей. Тогда \(k=5, h = \frac{b-a}{k}=\frac{1}{5}=0.2\).
	\begin{center}
		\begin{tabular}{|c|c|c|c|c|c|c|}
			\hline
			\(x\)& \(0\)& \(0.2\)& \(0.4\)& \(0.6\)& \(0.8\)& \(1\)\\\hline
			\(y\)& \(1\)& \(0.960789\)& \(0.852144\)& \(0.697676\)& \(0.527292\)& \(0.367879\)\\
			\hline
		\end{tabular}
	\end{center}
	\newpage
	Составим матрицу \(W\) и вектор \(F\) по формулам (\ref{matrixEq}) и (\ref{matrixW}).
	\begin{equation}
		\label{Wtest}
		W=
		\begin{pmatrix}
			2.9& -2.5& 0& 0& 0& 0\\
			0.92& -1.8736& 1.08& 0& 0& 0\\
			0& 0.84& -1.8544& 1.16& 0& 0\\
			0& 0& 0.76& -1.8224& 1.24& 0\\
			0& 0& 0& 0.68& -1.7776& 1.32\\
			0& 0& 0& 0& 0& 0.6
		\end{pmatrix}
	\end{equation}
	\begin{equation}
		\label{Ftest}
		F=
		\begin{pmatrix}
			0.4;& 0.038432;& 0.034086;& 0.027907;& 0.021092;& 0.220728
		\end{pmatrix}
	\end{equation}
	Решая уравнение \(WY=F\) методом прогонки, получаем вектор
	\begin{equation}
		\label{Ytest}
		Y=
		\begin{pmatrix}
			0.899895;& 0.883878;& 0.802373;& 0.672024;& 0.518387;& 0.367879
		\end{pmatrix}
	\end{equation}
	Теперь построим график точной функции и полученной с помощью метода конечных разностей, найдем ошибки вычислений.
	\begin{center}
		\includegraphics{test}
	\end{center}
	Итого мы видим, что графики достаточно близки, при этом правый конец отрезка точно совпадает с функцией, а левый --- нет. Это объясняется тем, что \(\beta_1=0, \alpha_1 \neq 0\). То есть, нет производной на правом конце.
	Ошибка вычисления:
	\begin{center}
		\begin{tabular}{|c|c|c|c|c|c|c|}
			\hline
			\(x\)& \(0\)& \(0.2\)& \(0.4\)& \(0.6\)& \(0.8\)& \(1\)\\\hline
			\(\varepsilon\)& \(0.100105\)& \(0.076911\)& \(0.049771\)& \(0.025652\)& \(0.008905\)& \(0\)\\
			\hline
		\end{tabular}
	\end{center}
	Бесконечная норма вектора погрешности равна \(0.100105\).
	\newpage
	\chapter{Контрольные тесты}
	Для исследования МКР 1-го порядка проведем следующие тесты:
	\begin{enumerate}
		\item Построим графики точного и численного решения.
		\item Построим график ошибки.
		\item Исследуем зависимость погрешности от \(h\). Сравним с теоретической ошибкой.
		\item Исследуем влияние ошибок в начальных данных.
	\end{enumerate}
	\chapter{Модульная структура программы}
	\begin{itemize}
		\item double q(double x) --- функция для вычисления \(q(x)\) в заданной точке.
		\item double r(double x) --- функция для вычисления \(r(x)\) в заданной точке.
		\item double f(double x) --- функция для вычисления \(f(x)\) в заданной точке.
		\item void FillBoard(int n, const double* X, double h, double* w1, double* w2, double* w3) --- функция для заполнения трехдиагональной матрицы.
		\item double* FillF(const double* const X, int n, double h) --- функция для заполнения вектора \(F\) свободных членов.
		\item void SolveMatrix(int n, double* a, double* c, double* b, double* f, double* x) --- функция, решающая матрицу методом прогонки.
	\end{itemize}
	\chapter{Численный анализ}
	\section{Графики функций}
	Построим графики точного и численного решения для \(n=50\) и \(n=20\).
	\begin{center}
		\includegraphics{func}
	\end{center}
	Мы можем видеть, что график численного решения приближается к графику точного решения с увеличением \(n\). Особенно это заметно в сравнении с графиком из тестового примера.
	\newpage
	\section{График ошибки}
	Для решений из предыдущего пункта построим графики ошибок.
	\begin{center}
		\includegraphics{error}
	\end{center}
	Заметим, что ошибка сходится к нулю на правом отрезке, что также было видно на тестовом примере. Там же находится обоснование такого поведения графика ошибки.
	\section{Зависимость погрешности от \(h\)}
	Будем изменять количество разбиений \(n\) от 500 до 10 и построим график бесконечной нормы ошибки от \(h\).
	\begin{center}
		\includegraphics{errorh}
	\end{center}
	В данной работе был реализован метод первого порядка. Согласно ожиданиям и теории, при уменьшении \(h\) на один порядок, ошибка так же уменьшается на один порядок, что видно на графике.
	\section{Влияние ошибок в начальных данных}
	Внесем возмущение разных порядков (от \(10^{-14}\) до \(10^{-1}\)) в числа \(A\) и \(B\) как уменьшая их, так и увеличивая.
	\begin{center}
		\includegraphics[scale=0.55]{outrageerror}
		\includegraphics[scale=0.55]{outragefunc}
	\end{center}
	По этим двум графикам мы видим, что метод устойчив к возмущениям малых порядков (до \(10^{-5}\)). Затем погрешность становится больше, если делать отрицательное возмущение и меньше, если возмущение положительное, но только до некоторого момента (\(10^{-3}\)). Значения \(A, B\) напрямую связаны с \(y(a), y(b)\). Если рассматривать конкретно данный вариант условия, то \(A=y(a), B=y(b)\). Рассматривая графики из тестового примера и пункта (7.1), заметим, что численное решение приближается к точному снизу. Поэтому, когда мы вносим отрицательное возмущение, мы "опускаем" график численного решения. Следовательно, погрешность увеличивается. А если вносить положительное возмущение, график "поднимается", погрешность уменьшается. Но только до тех пор, пока график численного решения не пересечет график точного решения. Как только график с положительным возмущением становится выше точного графика функции, погрешность начинает увеличиваться.
	\chapter{Общие выводы}
	По результатам исследования, можно отметить следующее. Данный метод требует достаточно большого объема вычислений, так как даже визуально графики начинают совпадать только при нескольких сотнях отрезков разбиения, а значит и матрица для решения данной краевой задачи будет иметь большие размеры, что негативно скажется на затратах памяти, даже если хранить только значения трех ее диагоналей. Следовательно, метод сходится медленно, имея достаточно большую ошибку. Зависимость выражается как \(\mathcal{O}(h)\). Кроме того, метод плохо устойчив к возмущениям начальных данных. Ошибка существенно возрастает уже при возмущении около \(10^{-5}\). Однако, в зависимости от конкретной данной функции возмущение может улучшить результат. Но логичнее это назвать удачей, чем способом достижения результата.
\end{document}